\documentclass[a4paper]{article}

\usepackage{comment}
\usepackage{amssymb}
\usepackage{amsthm}
\usepackage{lipsum}
\usepackage{amsmath}
\usepackage{url}
\usepackage{xcolor}
\usepackage{graphicx}
\usepackage{geometry}
\usepackage{enumerate}
\usepackage{hyperref}
\usepackage{epsfig}
\usepackage[linesnumbered]{algorithm2e}

\newcommand{\QMark}{{\color{red} \large ???}}

\geometry{left=3.17cm,right=3.17cm,top=2.54cm,bottom=2.54cm}

% https://en.wikibooks.org/wiki/LaTeX/Fonts
\renewcommand{\rmdefault}{ppl}

\newtheorem{definition}{Definition}[section]
\newtheorem{remark}{Remark}[section]
\newtheorem{proposition}{Proposition}[section]
\newtheorem{theorem}{Theorem}[section]
\newtheorem{lemma}{Lemma}[section]
\newtheorem{example}{Example}[section]
\newtheorem{corollary}{Corollary}[section]

\title{My Template}
\author{Jiannan Yang}

\begin{document}
\maketitle

\begin{comment}
\begin{abstract}
This is abstract.
\\
\textbf{Keyword:} abc, def
\end{abstract}
\end{comment}

\section{Theorem and Formula}
This is citation \cite{fan2013making,dwork2012privacy,gray2013consensus,cormen2009introduction}.
This is footnote \footnote{\url{http://www.songhuadan88.com}}.

\begin{definition}
This is definition.
\end{definition}


\begin{theorem}
\label{theorem}
This is theorem.
\end{theorem}

\begin{proof}
This is proof.
\begin{equation}
\label{equation}
1+1=2\mid{}2+2=4.
\log {\rm polylog}\ \text{abc}
\end{equation}
\end{proof}

\section{Equation}
All equations share the same label.
\begin{equation}
\begin{split}
1+1&+1 \\
2&+2+2 \\
3+3&+3 \\
4&+4+4
\end{split}
\end{equation}
\\
Each equation has its label.
\begin{align}
1+1&=1 \\
2&=2+2 \\
1&=3+4 \\
4&=6+7
\end{align}

\section{Reference}
Equation~\eqref{equation}
Theorem~\ref{theorem}
Section~1
Figure~1.1

\section{Itemize}
A list with predefined item title.
\begin{enumerate}[$(1)$]
\item First Item
\item Second Item
\item Third Item
\end{enumerate}

\section{Table}
The second column is fixed to 5cm.
\begin{table}[!ht]
\renewcommand\arraystretch{1.4} % set row spacing, only work for this table
\begin{center}
\caption{Oh, my table}
\label{table}
\begin{tabular}{|c|p{5cm}|}
\hline
cell11 & cell12 \\
\hline
cell21 & cell22 \\
\hline
cell31 & cell32 \\
\hline
\end{tabular}
\end{center}
\end{table}

\section{Figure}
Use trim to adjust figure and width to resize.
\begin{figure}[!ht]
  \centering
  \fbox{\includegraphics[trim=0cm 0cm 0cm 0cm,width=5cm]{icon.png}}
%   width=\textwidth
  \caption{caption}\label{fig:something}
\end{figure}

\section{Algorithm}
See Alg.~\ref{alg:something}, Line~\ref{code:step}.

 \begin{algorithm}[!ht] % use [H] if in beamer
  \KwData{The first parameter; the second parameter; the third parameter;}
  \KwResult{The output}
  The first statement\;
  \label{code:step}
  $//$ Make some comment\;
  \For{something to traverse}
  {something in For\;}  
  \ForAll{something to traverse} % also ForEach
  {something in ForAll\;}
  \While{something to traverse}
  {something in While\;}
  \Repeat{condition to finish}
  {something in Repeat\;}
  \eIf{condition in If}
  {something in If\;}
  {something in Else\;}
  \Return{Something to return;}
  \caption{An Algorithm}
  \label{alg:something}
 \end{algorithm}

\section{Settings}
To eliminate output file on WinEdt:
Opinions, Execution Modes, Tex Options, Uncheck ``Use \%P if...''

\noindent However, I start to use overleaf. 

\bibliographystyle{abbrv} % abbrv,  alpha, unsrt
\bibliography{manuscript}
%\bibliography{C:/Users/username/Desktop/lab/bibfile}

\end{document}
